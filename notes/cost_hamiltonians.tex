\documentclass[10pt, oneside]{article} 
\usepackage{amsmath, amsthm, amssymb, calrsfs, wasysym, verbatim, bbm, color, graphics, geometry, hyperref, biblatex, mathtools}

\hypersetup{
	colorlinks=true,
	linkcolor=blue,
	urlcolor=blue
}

\addbibresource{ref.bib}

\geometry{tmargin=.75in, bmargin=.75in, lmargin=.75in, rmargin = .75in}  

\newcommand{\R}{\mathbb{R}}
\newcommand{\C}{\mathbb{C}}
\newcommand{\Z}{\mathbb{Z}}
\newcommand{\N}{\mathbb{N}}
\newcommand{\Q}{\mathbb{Q}}
\newcommand{\Cdot}{\boldsymbol{\cdot}}

\newtheorem{thm}{Theorem}
\newtheorem{defn}{Definition}
\newtheorem{conv}{Convention}
\newtheorem{rem}{Remark}
\newtheorem{lem}{Lemma}
\newtheorem{cor}{Corollary}
\newtheorem{prop}{Proposition}

\newcommand{\tr}{\mathrm{Tr}}


\title{QAOA Cost Hamiltonians}
\author{Jack Ceroni}
\date{August 2020}

\begin{document}

\maketitle
\tableofcontents

\vspace{.25in}

\section{Notes}

In this document, we list multiple cost Hamiltonians/optimization problems and outline why they 
may be beneficial to implement in PennyLane.

\subsection{Cost Hamiltonians}

\begin{itemize}
	\item \textbf{Travelling Salesman}
	\item \textbf{MaxClique}
	\item \textbf{Max $k$-SAT}
	\item \textbf{MinGraphColouring}
	\item 
\end{itemize}

\end{document}
