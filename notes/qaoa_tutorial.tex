\documentclass[10pt, oneside]{article} 
\usepackage{amsmath, amsthm, amssymb, calrsfs, wasysym, verbatim, bbm, color, graphics, geometry, hyperref, biblatex, mathtools}

\hypersetup{
	colorlinks=true,
	linkcolor=blue,
	urlcolor=blue
}

\addbibresource{ref.bib}

\geometry{tmargin=.75in, bmargin=.75in, lmargin=.75in, rmargin = .75in}  

\newcommand{\R}{\mathbb{R}}
\newcommand{\C}{\mathbb{C}}
\newcommand{\Z}{\mathbb{Z}}
\newcommand{\N}{\mathbb{N}}
\newcommand{\Q}{\mathbb{Q}}
\newcommand{\Cdot}{\boldsymbol{\cdot}}

\newtheorem{thm}{Theorem}
\newtheorem{defn}{Definition}
\newtheorem{conv}{Convention}
\newtheorem{rem}{Remark}
\newtheorem{lem}{Lemma}
\newtheorem{cor}{Corollary}
\newtheorem{prop}{Proposition}

\newcommand{\tr}{\mathrm{Tr}}


\title{QAOA Tutorial Notes}
\author{Jack Ceroni}
\date{August 2020}

\begin{document}

\maketitle
\tableofcontents

\vspace{.25in}

\section{Notes}

The following notes describe the structure of the tutorial for the PennyLane QAOA module:

\begin{enumerate}
	\item The tutorial should begin by discussing very briefly what QAOA is, but without diving into very much depth 
		(for users who want a more detailed explanation, we can refer them to the other QAOA tutorial)
	\item The problem we solve in this QAOA tutorial should probably be something other than MaxCut (as that was already done 
		in the other QAOA tutorial). Maybe TSP?
	\item Assuming we do choose to do TSP, the next section could talk about how, to do QAOA, we need cost and mixer Hamiltonians 
		to define cost and mixer layers. We show that building the cost Hamiltonian and mixer Hamiltonians for TSP is complicated, then show how it 
		can be done with one line with PL QAOA.
	\item Explain how we can pass these cost and mixer Hamiltonians into qaoa.cost and qaoa.mixer to create the QAOA layers, and then 
		pass them into qml.repeat to create the ansatz.
	\item Show how we can easily add these features into a standard PL workflow.
	

	
\end{enumerate}

\end{document}
