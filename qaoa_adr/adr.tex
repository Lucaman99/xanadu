\documentclass{article}
\usepackage{inputenc, geometry, hyperref, amsmath, biblatex, amsfonts, minted}
\geometry{tmargin=.75in, bmargin=.75in, lmargin=.85in, rmargin = .85in}

\addbibresource{ref.bib}

\begin{document}

\section{Add QAOA Functionality to PennyLane}

\subsection{Status}

\subsection{Context}

The Quantum Approximate Optimization Algorithm is a variational quantum algorithm for solving combinatorial 
optimization problems, introduced by Farhi, Gutmann and Goldstone in 2014 [1].
\newline\newline
\noindent
To begin, we consider a cost function of the form:

\begin{equation}
	C(z) \ = \ \displaystyle\sum_{n} C_n(z)
\end{equation}
\noindent
where $z$ is some bitstring, and each $C_n(z)$ is a \textit{clause}, which is a function that maps 
each bitstring to either $0$ or $1$. Our goal is to minimize this cost function by finding the 
optimal bitstring(s) that return the lowest value.
\newline\newline
\noindent
To solve this problem with QAOA, we first map the cost function 
to a diagonal cost Hamiltonian, where the expectated energy with respect to 
each basis state is its corresponding cost function value when expressed as a bitstring 
(for instance, $|000\rangle$ corresponds to the bitstring $000$).
\newline\newline
\noindent
Next, we use a variational ansatz to prepare a parametrized quantum state.
Beginning with some initial state, $|\psi_0\rangle$ (usually an 
even superposition over all basis states), we evolve this state to get $|\psi(\gamma, \alpha)\rangle$ with an 
"alternating operator ansatz", similar to a Trotterized version of adiabatic quantum evolution:

$$|\psi(\gamma, \alpha)\rangle \ = \ \displaystyle\prod_{n \ = \ 1}^{P} e^{-i \alpha_n H_M} e^{-i \gamma_n H_C} |\psi_0\rangle$$
\noindent
where $H_C$ is the cost Hamiltonian that we are attemtping to minimize, and $H_M$ is a non-commuting mixer Hamiltonian.
\newline\newline
\noindent
Upon execution of this circuit, we calculate the quantity $F(\gamma, \alpha) \ = \ \langle \psi(\gamma, \alpha) | H_C | \psi (\gamma, \alpha) \rangle$.
We repeat this process many times, minimizing the value of $F(\gamma, \alpha)$.
This in turn minimizes the cost function and
gives us an optimal circuit that produces the optimal bitstring with high probability.
Finally, the optimal circuit is executed multiple times and the resulting bitstring that 
yields the lowest value of the cost function is outputted.
\newline\newline
\noindent
QAOA has the potential to generate approximate solutions to many well-known and important optimization problems 
such as QUBO, Max-k-SAT, Machine Scheduling, and more. The possibility of interesting applications, combined 
with a variational ansatz that is relatively easy to implement, compared to other instances of VQE, makes QAOA 
a very attractive candidate for implementation on NISQ devices, and possibly even a demonstration of quantum supremacy.
\newline\newline
\noindent 
In recent years, quantum researchers have become increasingly
interested in performing benchmarking and simulations of different versions of QAOA on classical computers [2-4], making the 
addition of a QAOA module to PennyLane a potentiallly valuable tool for this kind of work. In addition, QAOA has received a lot 
of attention from the community of quantum enthusiasts. A PennyLane QAOA module would be a fantastic learning resource
for many people trying to learn more about variational quantum algorithms, and may serve to attract more people to the entire PennyLane library.


\subsection{Analysis}

There are wour six components that are necessary to defining a QAOA workflow:

\begin{enumerate}
	\item Specifying the optimization problem
	\item Allowing the user to define a state initialization circuit
	\item Allowing the user to define a mixer layer
	\item Creating a cost layer from the chosen optimization problem and combining it with the mixer and an initialization circuit 
	      to create a variational circuit
      	\item Optimization of the variational circuit
	\item Post-processing
\end{enumerate}
\noindent
The objective of this ADR will disucss each in detail, and outline how they will be implemented in 
a PennyLane QAOA module.
\newline\newline
The QAOA module should be as \textbf{problem-centric} as possible. This means that 
the user should be able to define a QAOA workflow for some optimization problem without 
worrying too much about the exact specifics of the quantum circuits in the backend.

\subsection{Location in PennyLane}

We propose the creation of a new QAOA module where all of the functionality 
for creating QAOA workflows would be located.

\subsection{Specification of the Optimization Problem}

The main component of any QAOA procedure is the cost function that the user 
chooses to optimize. In the QAOA module, we propose that the user have 
access to a variety of built-in QAOA problem instances.
To start, we suggest that the following optimization problems be implemented:

 \begin{itemize}
	 \item MaxCut (for a user-defined graph)
	 \item QUBO (for a user-defined Ising model)
\end{itemize}

After the core functionality has been built, more common optimization problems can be added 
to the library.
\newline\newline
\noindent
Each of these built-in optimization problems will
be instances of this "Problem" class with the following attributes:

   \begin{itemize}

	   \item The cost Hamiltonian, implemented as an instance of the pennylane Hamiltonian
		  class.
	   \item The cost layer, which can be constructed by simply exponentiating each term of the cost 
		   Hamiltonian. Converting the cost Hamiltonian into a differentiable cost unitary is thus 
		   very straightforward, and does not require any complicated decomposition.
	   \item A \textit{recommended} mixer layer, which is a function containing
                 PennyLane operations. The default is simple RX layer used in the original QAOA paper.
	   \item A \textit{recommended} initialization circuit, which is also a function 
                 containg PennyLane operations. Generally, this will either be an 
   		 even superposition over all bitstrings, or an even superposition over all 
   		 bitstrings satisfying a certain constraint (for example, Hamming weight 1).
   		 The default value is a method creating an even superposition.
   \end{itemize}

\noindent
Consider an example of a built-in problem instance: MaxCut. The cost Hamiltonian will be defined as:

\begin{equation}
	H_C \ = \ \displaystyle\sum_{(i, j) \in E(G)} \frac{1}{2} \big(1 \ - \ Z_i Z_j \big)
\end{equation}

for a user inputted graph $G$. Thus, the cost Hamiltonian attribute of 


\noindent
\textbf{Note:} The user isn't required to use either the recommended mixer, nor the
initialization circuit, they are merely suggestions as to what works best 
for each built-in problem

\subsubsection{Further Notes on Mixer Layers}

There are a wide variety of mixers that can be used for QAOA. A good mixer 
Hamiltonian should be non-commuting with the cost and constrain the search space of different bitstrings 
to those that satisfy a certain \textit{hard-constraint} (for instance, Hamming weight 1). As a result, the correct 
choice of a mixer can vary greatly between different QAOA workflows.
\newline\newline
\noindent
Since the user is not constrained to using the recommended mixer that is built-into each 
problem instance, we must specify convenient ways for a user to utilize a variety of different 
mixers in their QAOA workflows.
\newline\newline
\noindent
Similar to defining a cost Hamiltonian, we let the user either: 
\begin{itemize}
	\item Define their own method 
	      containing PennyLane operations, which 
	      can be used as the mixer layer. 
	\item Choose from a variety of built-in mixers.
              To begin, we propose the following mixer layers are implemented:

		\begin{itemize}
			\item The $RX$ mixer
			\item The $XY$ mixer
			\item The Hamming weight-preserving generalized SWAP mixer, defined as:
				$$U_M \ = \ \displaystyle\prod_{(i, j) \in G} \ \exp \Big[ - \frac{\beta}{2} (X_{i} X_{j} \ + \ Y_{i} Y_{j}) \Big]$$
		\end{itemize}
\end{itemize}
\noindent
\textbf{Note:} Each of the recommended mixers in each problem instance will simply be a call 
to one of the built-in mixers defined above, with the default for all problem instances
being the $RX$ mixer.


\subsubsection{Defining the QAOA Circuit}

With all of the necessary peices of the QAOA defined, they must be combined 
to create a variational QAOA circuit. We outline two possible ways to do this.
\newline\newline
\noindent
The first option is to have the user take their cost and mixer layers, along 
with some kind of qubit initialization, and pass it into a circuit 
like this:

\begin{minted}
        [
    frame=lines,
    framesep=2mm,
    baselinestretch=1.2,
    bgcolor=white,
    fontsize=\footnotesize,
    ]
    {python}
    
    import pennylane as qml
    from pennylane import qaoa
    
    qubits = range(4)
    dev = qml.device("default.qubit", wires=len(qubits))
    
    # Defines the graph and the cost Hamiltonian for MaxCut
    
    graph = [(0, 1), (1, 2), (2, 3), (3, 0)]
    maxcut = qaoa.problems.maxcut(graph=graph)
    
    # Defines the unitaries, the circuit and the cost function
    
    initialization = maxcut.init
    cost_layer = maxcut.cost
    mixer_layer = maxcut.mixer

    qaoa_circuit = qaoa.circuit(cost_layer, mixer_layer, depth=3)
    
    def circuit(params):
    	
	initialization()
        qaoa_circuit(params)
        
        return qml.expval(qml.Hermitian(maxcut.hamiltonian, wires=qubits))
    
    # Defines the cost function
    
    cost_function = qml.QNode(circuit, dev)

   \end{minted}

\noindent
The benefit of this method is that it allows for more 
flexibility in what goes on inside the variational circuit and
the value that is returned after execution. It does, however, place 
more of a burden on the user to code out the whole variational circuit 
themself.
\newline\newline
\noindent
The second option is to have something like this:

\begin{minted}
        [
    frame=lines,
    framesep=2mm,
    baselinestretch=1.2,
    bgcolor=white,
    fontsize=\footnotesize,
    ]
    {python}
    
    import pennylane as qml
    from pennylane import qaoa
    
    qubits = range(4)
    dev = qml.device("default.qubit", wires=len(qubits))
    
    # Defines the graph and the cost Hamiltonian for MaxCut
    
    graph = [(0, 1), (1, 2), (2, 3), (3, 0)]
    maxcut = qaoa.problems.maxcut(graph=graph)
    
    # Defines the unitaries, the circuit and the cost function
    
    mixer_layer = maxcut.mixer
    init_layer = maxcut.init
    
    qaoa_circuit = qaoa.circuit(maxcut, mixer_layer, init_layer, depth=3)
    
    # Defines the cost function
    
    cost_function = qaoa.sampling_cost(qaoa_circuit, samples=200)

    \end{minted}

\noindent
This allows for the user to easily define the variational QAOA
circuit with a couple lines of code, but without as much customizability as the former 
option.
\newline\newline
\noindent
\textbf{Note:} The reason that we choose to keep the cost function separate from the 
QAOA circuit in this option is due to the fact that the QAOA circuit (not necessarily 
the cost function) is usually needed for post-processing.
\newline\newline
\noindent
\textbf{Conclusion:} The second option would probably be best for users. In the case that 
someone wanted as much freedom and customizability as the first option provides, they 
may as well code the QAOA circuit completely manually, without the help of the module. 
The second option on the other hand is still customizeable, but it allows the user 
to define and run a circuit in far fewer lines of code, and with much greater ease.

\subsubsection{Post-processing}

The final step of any QAOA workflow is post-processing of the 
data after determining the parameters that yield the optimal 
probability distribution of bitstrings. The goal of the QAOA is 
to find the bitstring that optimizes the cost function. Thus, we 
introduce functionality that can be used to determine this value, as well as 
other properties of the outputted data.
\newline\newline
\noindent
Within the core functionality of the PennyLane QAOA module, 
there are two post-processing functions that we propose:

\begin{itemize}
	\item A function that samples from the variational QAOA circuit, for 
		a given set of parameters. This allows the user to sample from the 
		circuit many times, and infer things like the most commonly occurring 
		bitstring, the average value of the cost function, the standard deviation 
		of samples, etc.
	\item A function that returns the probability of measuring a given bitstring after executing 
		a given QAOA circuit for some collection of parameters. This would allow the users to validate 
		that their QAOA experiment is working
		as expected, by ensuring that the bitstrings that correspond to optimal values of the cost 
		function occurr with high probability.
\end{itemize}

We choose to leave the post-processing functionality of the module fairly basic for the 
first iteration of the QAOA module, 
as a user that has some experience with Python should be able to take these two functions 
and use them to calculate essentially anything they want.

\section{Other}

There are a few other features that the QAOA library should likely include that don't fall into 
any of the previous categories:

\begin{itemize}
	\item Functions that initialize $\alpha$ and $\gamma$ parameters from normal and uniform distributions, 
		for a given circuit.
\end{itemize}

\section{User Interface}

All together, a general QAOA workflow might look something like this (assuming the second 
option for defining the QAOA circuit is chosen):

\begin{minted}
        [
    frame=lines,
    framesep=2mm,
    baselinestretch=1.2,
    bgcolor=white,
    fontsize=\footnotesize,
    ]
    {python}
    
    import pennylane as qml
    from pennylane import qaoa
    import math
    import numpy as np
    
    qubits = range(4)
    dev = qml.device("default.qubit", wires=len(qubits))
    
    # Defines a QUBO problem with the following structure:
    #   Interaction terms --> (qubit index 1, qubit index 2, weight)
    #   Bias terms --> (qubit index, bias)

    interaction_terms = [(0, 1, 2.7), (1, 2, 0.43), (2, 3, 1.2), (3, 0, 0.15)]
    bias_terms = [(0, 2.3), (3, 0.93)]

    ising_model = [interaction_terms, bias_terms]

    qubo = qaoa.problems.qubo(ising_model=ising_model)
    
    # Defines a custom mixer, using two built-in mixers
    def mixer_layer(beta):

	qaoa.mixers.XYLayer(beta, qubits, graph=[(0, 1), (1, 2), (2, 3), (3, 0)])
	qaoa.mixers.RXLayer(beta, qubits)

    # Defines the initialization circuit
    init_layer = qubo.init

    # Defines the QAOA circuit
    qaoa_circuit = qaoa.circuit(qubo, mixer_layer, init_layer, depth=3)

    # Defines a sampling cost function
    qaoa_cost = qaoa.sampling_cost(qubo, qaoa_circuit, samples=200)

    # Runs the circuit
    steps = 100
    optimizer = qml.AdamOptimizer()
    params = qaoa.uniform_params(qaoa_circuit)

    for i in range(0, steps):
    	params = optimizer.step(qaoa_cost, params)

    # Finds the optimal bitstring
    optimal_bitstring = None
    optimal_cost = math.inf
    samples = 100

    for i in range(0, samples):
    	sample = qaoa.sample(qaoa_circuit)
	val = np.vdot(sample, (qubo.hamiltonian @ sample))

	if val < optimal_cost:
	    optimal_cost = val
	    optimal_bitstring = sample
    
    print("The Optimal Bitstring is: {}".format(optimal_bitstring))

   \end{minted}

\printbibliography

\end{document}
