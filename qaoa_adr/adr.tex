\documentclass{article}
\usepackage{inputenc, geometry, hyperref, amsmath, biblatex, amsfonts, minted}
\geometry{tmargin=.75in, bmargin=.75in, lmargin=.85in, rmargin = .85in}

\addbibresource{ref.bib}

\begin{document}

\section{Add QAOA Functionality to PennyLane}

\subsection{Status}

\subsection{Context}

The Quantum Approximate Optimization Algorithm is a variational quantum algorithm for solving combinatorial 
optimization problems, introduced by Farhi, Gutmann and Goldstone in 2014 [1].
\newline\newline
\noindent
If we are given a cost function of the form:

\begin{equation}
	C(z) \ = \ \displaystyle\sum_{n} C_n(z)
\end{equation}
\noindent
where $z$ is some bitstring, and each $C_n(z)$ is a \textit{clause}, which is a function that maps 
each bitstring to either $0$ or $1$. Our goal is to minimize this cost function, by finding the 
bitstring(s) that returns the lowest value.
\newline\newline
\noindent
QAOA is able to solve this problem by taking a variational approach. We first map the cost function 
to a diagonal cost Hamiltonian. We then use a parametrized quantum circuit to repeatedly prepare ansatz states, 
measuring the expectation value of the cost Hamiltonian with respect to this state. This process is repeated with 
new parameters, chosen by a classical optimizer, until the algorithm converges on the state that yields the lowest 
energy expectation value. This yields a set of optimal parameters, which are used to execute the circuit many times to 
find the optimal bitstring(s), thus solving the optimization problem.
\newline\newline
\noindent
In QAOA, the ansatz state is prepared in a unique way. Beginning with some initial state, $|\psi_0\rangle$ (usually an 
even superposition over all basis states), we evolve this state to get $|\psi(\gamma, \alpha)\rangle$ with the 
alternating unitary evolution, similarly to a Trotterized version of adiabatic quantum evolution:

$$|\psi(\gamma, \alpha)\rangle \ = \ e^{-i \alpha_P H_M} e^{-i \gamma_P H_C} \ ... \  e^{-i \alpha_2 H_M} e^{-i \gamma_2 H_C}  e^{-i \alpha_1 H_M} e^{-i \gamma_1 H_C} |\psi_0\rangle$$
\noindent
where $H_C$ is the cost Hamiltonian that we are attemtping to minimize, and $H_M$ is a non-commuting mixer Hamiltonian.
\newline\newline
\noindent
The cost function is then defined to be $F(\gamma, \alpha) \ = \ \langle \psi(\gamma, \alpha) | H_C | \psi (\gamma, \alpha) \rangle$, which 
we minimize, as was described above.
\newline\newline
\noindent
QAOA has the potential to generate approximate solutions to many well-known and important optimization problems 
such as QUBO, Max-k-SAT, Machine Scheduling, and more. The possibility of interesting applications, combined 
with a variational ansatz that is relatively easy to implement, compared to other instances of VQE, makes QAOA 
a very attractive candidate for implementation on NISQ devices. Extensive research 
has been done on the possibility of using it as a robust variational algorithm that can 
achieve quantum supremacy on near-term quantum computers and as 
a result, researchers are interested in performing calculations, simulations, and benchmarking of different 
QAOA experiments on classical computers.


\subsection{Analysis}

There are four main components that are necessary to defining a QAOA workflow:

\begin{enumerate}
	\item Specifying the optimization problem
	\item Allowing the user to define a mixer layer
	\item Creating a cost layer and combining it with the mixer and pre-processing 
	      to create a variational circuit
      	\item Post-processing of results 
\end{enumerate}
\noindent
This ADR will disucss each in detail.
\newline\newline
The main goal of the QAOA module is to make it as \textbf{problem-centric} as possible, 
where the user can define a QAOA workflow for some optimization problem without 
worrying too much about the exact specifics of the quantum circuits in the backend.

\subsubsection{The Cost and Mixer Layers}

The main component of any QAOA procedure is the cost function that the user 
wants to optimize. In the PennyLane QAOA module, we propose two ways that a user 
can go about choosing an optimization problem to solve:

\begin{enumerate}
	\item The user passes a custom cost Hamiltonian matrix into the workflow.
	\item The user chooses from one of the built-in cost Hamiltonians, which 
              correspond to well-known optimization problems. To begin, we propose 
   	      that the following optimization problems be implemented:
	      \begin{itemize}
		      \item MaxCut (for a user-defined graph)
		      \item QUBO (for a user-defined Ising model)
		\end{itemize}
\end{enumerate}
  \noindent 
   After the core functionality has been built, more common optimization problems can be added 
   to the library.
\newline\newline
\noindent
Each of these optimization problems will
be instances of this "Problem" class with the following attributes:

   \begin{itemize}

	   \item The matrix form of the cost Hamiltonian
	   \item The cost layer. With knowledge of the cost Hamiltonian, the cost 
   		 layer can be constructed. In theory, it is possible to simply 
		 take the cost Hamiltonian and decompose it into a linear combination of 
		 $Z$ gates, however, this process would lengthen the runtime of the simulation. 
		 Thus, when defining new, built-in cost function that have an already well-know 
		 quantum circuit implementation (for instance, MaxCut), there is an option to specify the circuit
		 directly.
	   \item A \textit{recommended} mixer layer, which is a function containing
                 PennyLane operations. The default is built-in RX layer.
	   \item A \textit{recommended} initialization circuit, which is also a function 
                 containg PennyLane operations. Generally, this will either be an 
   		 even superposition over all bitstrings, or an even superposition over all 
   		 bitstrings satisfying a certain constraint (for example, Hamming weight 1).
   		 The default value is an even superposition.
   \end{itemize}

\noindent
\textbf{Note:} The user isn't required to use either the recommended mixer, nor the
initialization circuit, they are merely suggestions as to what works best 
for each built-in problem
\newline\newline
\noindent
\textbf{Note:} The custom cost Hamiltonian will also be defined as an instance of the of
this general problem class, with the default values for the mixer and initialization 
attributes. 
\newline\newline
\noindent
There are also cases to consider where the circuit decomposition of the cost layer corresponding to some cost 
Hamiltonian is unknown (for example, when the user passes in their own custom Hamiltonian
). Thus, we must write a subroutine that performs the decomposition for us. We begin by expressing the cost Hamiltonian 
as a linear combination of Pauli-Z gates. Since the Pauli gates form an orthogonal
basis of the vector space of linear transformations, we will have a unique decomposition
of our Hamiltonian $H_C$ in terms of $Z$ gates. In fact, we will have:

$$H_C \ = \ \displaystyle\sum_{n} c_n Z_1^{n_1} \otimes Z_2^{n_2} \otimes \ ... \ \otimes Z_{k}^{n_k}$$

\noindent
with $n_j \in \{0, \ 1\}$. It follows that:

$$c_n \ = \ \text{Tr}( H_C Z_1^{n_1} \otimes Z_2^{n_2} \otimes \ ... \ \otimes Z_{k}^{n_k})$$

\noindent
This means that with knowledge of $H_C$, its full decomposition in terms of powers 
of $Z$ gates can be calculated. Since each term of the Hamiltonian commutes, we 
will have as the cost unitary:

$$U_C \ = \ \displaystyle\prod_{n} e^{- i \gamma c_n Z_1^{n_1} \otimes Z_2^{n_2} \otimes \ ... \ \otimes Z_{k}^{n_k}}$$

\noindent
Each exponentiated $Z$ in this unitary can be implemented as a MultiRZ gate in
PennyLane, and is thus differentiable.

\subsubsection{Further Notes on Mixer Layers}

There are a wide variety of mixers that are suited to different instances of QAOA.
Since the user is not constrained to using the recommended mixer that is built-into each 
problem instance, we must specify convenient ways for a user to utilize a variety of different 
mixers in their QAOA workflows.
\newline\newline
\noindent
Similar to defining a cost Hamiltonian, we let the user either: 
\begin{itemize}
	\item Define their own function, 
	      containing PennyLane operations, which 
	      can be used as the mixer layer. 
	\item Choose from a variety of built-in mixers.
              To begin, we propose the following mixers \textbf{unitaries} are implemented:

		\begin{itemize}
			\item The $RX$ mixer
			\item The $XY$ mixer
			\item The Hamming weight-preserving generalized SWAP mixer, defined as:
				$$\exp \Big[ - \frac{\beta}{2} (X_{i} X_{j} \ + \ Y_{i} Y_{j}) \Big]$$
		\end{itemize}
\end{itemize}
\noindent
\textbf{Note:} Each of the recommended mixers in each problem instance will simply be a call 
to one of the built-in mixers defined above, with the default being the $RX$ mixer.


\subsubsection{Defining the QAOA Circuit}

With all of the necessary peices of the simulation defined, it must be combined 
to create a variational QAOA circuit. We see two possible ways to do this.
\newline\newline
\noindent
The first option is to have the user take their cost and mixer layers, along 
with some kind of qubit initialization, and pass it into a circuit 
like this:

\begin{minted}
        [
    frame=lines,
    framesep=2mm,
    baselinestretch=1.2,
    bgcolor=white,
    fontsize=\footnotesize,
    ]
    {python}
    
    import pennylane as qml
    from pennylane import qaoa
    
    qubits = range(4)
    dev = qml.device("default.qubit", wires=len(qubits))
    
    # Defines the graph and the cost Hamiltonian for MaxCut
    
    graph = [(0, 1), (1, 2), (2, 3), (3, 0)]
    maxcut = qaoa.problems.maxcut(graph=graph)
    
    # Defines the unitaries, the circuit and the cost function
    
    initialization = maxcut.init
    cost_layer = maxcut.cost
    mixer_layer = maxcut.mixer

    qaoa_circuit = qaoa.circuit(cost_layer, mixer_layer, depth=3)
    
    def circuit(params):
    	
	initialization()
        qaoa_circuit(params)
        
        return qml.expval(qml.Hermitian(maxcut.hamiltonian, wires=qubits))
    
    # Defines the cost function
    
    cost_function = qml.QNode(circuit, dev)

   \end{minted}

\noindent
The benefit of this method is that it allows for more 
flexibility in what goes on inside the variational circuit and
the value that is returned after execution. It does, however, place 
more of a burden on the user to code out the whole variational circuit 
themself.
\newline\newline
\noindent
The second option is to have something like this:

\begin{minted}
        [
    frame=lines,
    framesep=2mm,
    baselinestretch=1.2,
    bgcolor=white,
    fontsize=\footnotesize,
    ]
    {python}
    
    import pennylane as qml
    from pennylane import qaoa
    
    qubits = range(4)
    dev = qml.device("default.qubit", wires=len(qubits))
    
    # Defines the graph and the cost Hamiltonian for MaxCut
    
    graph = [(0, 1), (1, 2), (2, 3), (3, 0)]
    maxcut = qaoa.problems.maxcut(graph=graph)
    
    # Defines the unitaries, the circuit and the cost function
    
    mixer_layer = maxcut.mixer
    init_layer = maxcut.init
    
    qaoa_circuit = qaoa.circuit(maxcut, mixer_layer, init_layer, depth=3, output="expval")
    
    # Defines the cost function
    
    cost_function = qml.QNode(qaoa_circuit, dev)

    \end{minted}

\noindent
This allows for the user to easily define the variational QAOA
circuit with one line of code, but without as much customizability as the former 
option.

\subsubsection{Post-processing}

The final step of any QAOA workflow is post-processing of the 
data, after determining the parameters that yield the optimal 
probability distribution of bitstrings. The goal of the QAOA is 
to find the bitstring that optimizes the cost function, thus, we 
introduce functionality that can determine this value, as well as 
other properties of the outputted data.
\newline\newline
\noindent
Within the core functionality of the PennyLane QAOA module, 
there are two post-processing functions that we propose:

\begin{itemize}
	\item A function that samples from the variational QAOA circuit, for 
		a given set of parameters. This allows the user to sample from the 
		circuit many times, and infer things like the most commonly occurring 
		bitstring, the average value of the cost function, the standard deviation 
		of samples, etc.
	\item A function that returns the probability of measuring a given bitstring after executing 
		a given circuit. This would allow the users to validate that their QAOA experiment is working
		as expected, by ensuring that the bitstrings that correspond to optimal values of the cost 
		function occurr with high probability.
\end{itemize}

\printbibliography

\end{document}
